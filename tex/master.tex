\documentclass[a4paper, 11pt]{report}
\usepackage[T1]{fontenc}
\usepackage{lmodern}
\usepackage[polish]{babel}
\usepackage[utf8]{inputenc}
\usepackage{enumerate}
\usepackage{multirow}
\usepackage{graphicx}
\usepackage{caption}
%\usepackage{subcaption}
\usepackage{listings}
\usepackage{fancyhdr}
\usepackage{tikz}
\usepackage{latexsym}
\usepackage{subfig}
\usepackage{color}
\usepackage{pdflscape}
\pagestyle{fancy}
\usepackage[all,cmtip]{xy}
\selectlanguage{polish}
\title{\huge{Praca magisterska\\ ,,Bezprzewodowy system sterowania z wykorzystaniem systemu czasu rzeczywistego FreeRTOS''}}
\author{Jan Głos}
\date{26.06.2018}
\lhead{Praca magisterska} 																																						% określa lewą część nagłówka
\chead{} 																																															% określa środkową część nagłówka
\rhead{,,Bezprzewodowy system sterowania z wykorzystaniem systemu czasu rzeczywistego FreeRTOS''} 		% określa prawą część nagłówka
\lfoot{} 																																															% określa lewą część stopki
\cfoot{\thepage} 																																											% określa środkową część stopki
\rfoot{} 																																															% określa prawą część stopki
\newcommand{\ods}{\hspace*{1em}}

% ***@ DOCUMENT BEGIN @***

\begin{document}
\thispagestyle{empty}
\begin{figure}[t]
\centering
\includegraphics[width=14cm]{files/pk}
\end{figure}
\vspace{8cm}
\begin{Huge}
\begin{center}
\textsc{Wydział Inżynierii \\*Elektrycznej i Komputerowej}
\end{center}
\end{Huge}
\vspace{2cm}
\begin{center}
\begin{huge}
Praca magisterska\\ ,,Bezprzewodowy system sterowania z wykorzystaniem systemu czasu rzeczywistego FreeRTOS''\\
\end{huge}
\vspace{2cm}
\begin{LARGE}
Jan Głos\\

\vspace{5cm}
Promotor: \hfill Dr inż. Wojciech Mysiński
\hfill
\end{LARGE}
\end{center}

\newpage
\begin{small}
\tableofcontents
\end{small}


\newpage

% WSTĘP
\chapter{Wstęp}
\section{Rys historyczny}
\section{Aktualny stan wiedzy}
\section{Koncepcja pracy a aktualne metody}
% WSTĘP

% CEL PRACY
\chapter{Cel pracy}
\section{Cel}
\section{Założenia}

% CEL PRACY

% KONCEPT
\chapter{Koncepcja}
\section{Schemat blokowy}
% KONCEPT

% EFEKT
\chapter{Efekt końcowy na tle koncepcji}
\section{Pierwotne założenia projektu}
\section{Efekt końcowy}
% EFEKT

\chapter{Kierunki dalszych prac}

\chapter{Podsumowanie}

\begin{thebibliography}{99}


\end{thebibliography}



\end{document}
